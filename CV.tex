%% start of file `template.tex'.
%% Copyright 2006-2010 Xavier Danaux (xdanaux@gmail.com).
%
% This work may be distributed and/or modified under the
% conditions of the LaTeX Project Public License version 1.3c,
% available at http://www.latex-project.org/lppl/.


\documentclass[letter,10pt]{moderncv}

% moderncv themes
%\moderncvtheme[blue]{casual}                 % optional argument are 'blue' (default), 'orange', 'red', 'green', 'grey' and 'roman' (for roman fonts, instead of sans serif fonts)
\moderncvtheme[blue]{classic}                % idem

% character encoding
\usepackage[utf8]{inputenc}                   % replace by the encoding you are using
\usepackage{multicol}
\renewcommand{\refname}{Publicaciones}
% adjust the page margins
\usepackage[scale=0.85]{geometry}
%\setlength{\hintscolumnwidth}{3cm}						% if you want to change the width of the column with the dates
%\AtBeginDocument{\setlength{\maketitlenamewidth}{6cm}}  % only for the classic theme, if you want to change the width of your name placeholder (to leave more space for your address details
%\AtBeginDocument{\recomputelengths}                     % required when changes are made to page layout lengths

% personal data
\firstname{Ignacio}
\familyname{Villacura de la Paz}
\title{Ingeniero Civil Informático}               % optional, remove the line if not wanted
\address{San Guillermo 852, Dpto 404}{Valparaíso}    % optional, remove the line if not wanted
\mobile{+56 9 85294765}                    % optional, remove the line if not wanted
%\phone{phone (optional)}                      % optional, remove the line if not wanted
%\fax{fax (optional)}                          % optional, remove the line if not wanted
\email{ivillacu@inf.utfsm.cl}                      % optional, remove the line if not wanted
%\homepage{homepage (optional)}                % optional, remove the line if not wanted
%\extrainfo{additional information (optional)} % optional, remove the line if not wanted
%\photo[64pt]{Foto.jpg}                         % '64pt' is the height the picture must be resized to and 'picture' is the name of the picture file; optional, remove the line if not wanted
%\quote{Some quote (optional)}                 % optional, remove the line if not wanted

% to show numerical labels in the bibliography; only useful if you make citations in your resume
\makeatletter
\renewcommand*{\bibliographyitemlabel}{\@biblabel{\arabic{enumiv}}}
\makeatother

% bibliography with mutiple entries
%\usepackage{multibib}
%\newcites{book,misc}{{Books},{Others}}

%\nopagenumbers{}                             % uncomment to suppress automatic page numbering for CVs longer than one page
%----------------------------------------------------------------------------------
%            content
%----------------------------------------------------------------------------------
\begin{document}

\maketitle
\vspace{-30pt}
\begin{center}
    Mi principal interés actualmente está enfocado en las nuevas tecnologías,
    entretención digital y desarrollo web. 

%\emph{Gamification} y el desarrollo
 %   de videojuegos son los temas que me atraen actualmente.
 \end{center}
\section{Datos Personales}
\cvcomputer{Rut}{16.792.342-9}
           {Nacimiento}{19 de Enero 1988}
\cvcomputer{Estado Civil:}{Soltero}
           {Colegio:}{Colegio Concepción, Parral}
\cvcomputer{Universidad:}{Universidad Técnica Federico Santa María, Valparaíso}
           {}{}

\section{Experiencia Laboral}

\cventry{2008 -- 2014}
        {Ayudantías}
        {Departamento Informática}
        {UTFSM}
        {}
        {Laboratorio de Computación, Programación, Taller de Sistemas Computacionales.}

\cventry{2009 -- 2013}
        {Programador y Diseñador}
        {USMGames}
        {UTFSM}
        {Co-Fundador de USMGames, grupo de investigación y desarrollo de video
        juegos, parte del Programa de Iniciativas Estudiantiles Académicas (PIE>A).
        Mi participación se centró en el Diseño y Desarrollo de video juegos
        en los distintos proyectos de la iniciativa}
        {\url{http://usmgames.cl}, \url{http://www.piea.usm.cl}}

\cventry{2010}
        {Jefe de Proyecto}
        {PIE>A}
        {UTFSM}
        {}
        {A cargo del proyecto titulado: \emph{``Proyecto de investigación en ARM
        para la creación de micro-computadores''}, el cual realizó variados
        proyectos basados en sistemas ARM (BeagleBoard), desarrollando un
        reproductor multimedia, sistemas de información en tiempo real, etc.}

\cventry{2010 -- 2011}
        {Programador}
        {Brainstormers}
        {}
	{}
        {Ayuda en la programación de una novela gráfica en Renpy. También se diseñó
	un nuevo juego para la empresa.}

\cventry{2011 -- 2013}
        {Ayudante Curso de Programación en Python}
        {Laboratory of Interdisciplinary Research in Astro Engineering (LIRAE)}
        {UTFSM}
        {\url{http://www.lirae.cl}}
        {A cargo de la corrección de las actividades evaluadas de un curso
        de programación a miembros del proyecto Atacama Large Millimeter/submillimeter
        Array (ALMA).}

\cventry{2015}
        {Trainee}
        {Tinet Soluciones Informáticas S.A.}
        {Valparaíso}
        {\url{http://www.tinet.cl}}
        {Curso \emph{``Aplicación del lenguaje de programación java estandar''}. Curso JEE para aplicaciones
	empresariales.}

\subsection{Prácticas}

\cventry{2010 -- 2011}
        {Práctica industrial}
        {Mazorca Studios}
        {}
        {}
        {Investigación de Unreal Engine 3 para nuevos proyectos.}%

\cventry{2011 -- 2012}
        {Práctica profesional}
        {Guitar Boost SpA}
        {}
        {}
        {Programación y ayuda en el diseño del software Guitar Boost.}%

\section{Intereses}
\cvcomputer{Video Juegos}{Diseño y desarrollo. La utilización de video juegos 
	   en la vida cotidiana para acercar a la gente y entretenerla es por lo 
           que me interesa esta área de entretención digital. La utilización de 
	   nuevas tecnologías es lo que me atrae a esta área.}
           {Gamification}{La utilización de elementos de diseño de juegos en otros 
	   contextos. Esta es una técnica moderna para atraer y retener clientes 
           o usuarios utilizada mayormente en marketing y ventas. Mi interés es en
           la utilización de esta técnica para ayudar Pymes.}
\cvcomputer{E-commerce}{Implementación de sistemas de comercio electrónico
	   para que las empresas aumenten su exposición y ventas. Me interesan 
	   las interacciones con otros conceptos como gamification.}
	   {Desarrollo Web}{Las nuevas tecnologías van creando nuevas herramientas
	   para el desarrollo web. La integración con conceptos como gamification y
	   herramientas de uso masivo son interesantes.}

\section{Idioma}
\cvlanguage{Inglés}{Avanzado}{TOEIC $935$ $2012$}

\section{Computaci\'on}
\cvcomputer{Programaci\'on}{C, JEE, Python, PHP}{Servicios}{Git, SVN}
\cvcomputer{CMS}{Wordpress}{}{}

\section{Recomendaciones}
\cventry{}{Horst von Brand}
        {Profesor, Departamento de Informática}{UTFSM}{vonbrand@inf.utfsm.cl}{}
\cventry{}{Cecilia Reyes}
        {Profesora, Departamento de Informática}{UTFSM}{reyes@inf.utfsm.cl}{}

%\section{Pretensión de renta}
%\cventry{Renta liquida}{$\$850.000$}{}{}{}{}

% Publications from a BibTeX file without multibib\renewcommand*{\bibliographyitemlabel}{\@biblabel{\arabic{enumiv}}}% for BibTeX numerical labels
\nocite{*}                                                                           
\bibliographystyle{plain}                                                            
\bibliography{publicaciones}       % 'publications' is the name of a BibTeX file

\end{document}
